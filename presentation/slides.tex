\documentclass{beamer}

\usepackage{amsmath}
\usepackage{amsbsy}
\usepackage{bbm}

\def\MLine#1{\par\hspace*{-\leftmargin}\parbox{\textwidth}{\[#1\]}}
\usepackage[utf8]{inputenc}

\usepackage{utopia} %font utopia imported
\usepackage{lmodern}

\newcommand{\matr}[1]{\mathbf{#1}}
\newcommand{\norm}[1]{\left\lVert#1\right\rVert}
\newcommand{\E}[2][]{\mathbb{E}_{#1}\left\{#2\right\}}
\newcommand{\var}[2][]{var_{#1}\left\{#2\right\}}
\newcommand{\cov}[1]{cov\{#1\}} 
\newcommand{\normal}[1]{\mathcal{N}\left(#1\right)}
\newcommand{\exponents}[1]{exp\left\{#1\right\}}
\newcommand{\indicator}[1]{\mathbbm{1}_{#1}}

\newcommand{\bmu}{\boldsymbol{\mu}}
\newcommand{\bpi}{\boldsymbol{\pi}}
\newcommand{\bTheta}{\boldsymbol{\Theta}}
\newcommand{\bSigma}{\boldsymbol{\Sigma}}
\newcommand{\bphi}{\boldsymbol{\phi}}

\newcommand{\calA}{\mathcal{A}}
\newcommand{\calL}{\mathcal{L}} 
\newcommand{\calE}{\mathcal{E}}
\newcommand{\calR}{\mathcal{R}}
\newcommand{\calC}{\mathcal{C}}
\newcommand{\calD}{\mathcal{D}}
\newcommand{\bx}{\matr{x}}
\newcommand{\bt}{\matr{t}}
\newcommand{\bw}{\matr{w}}
\newcommand{\bX}{\matr{X}}
\newcommand{\bZ}{\matr{Z}} 
\newcommand{\bz}{\matr{z}}
\newcommand{\bu}{\matr{u}}
\newcommand{\bY}{\matr{Y}}

\usetheme{Madrid}
\usecolortheme{default}

\title[]{Monte Carlo Simulation on Credit Risk}
\author[Loora, Peiqi]{Loora Li \and Peiqi Wang \and Zeneng Fan \and Xiaoye Yuan \and Eris Zhang}
\date[July 5th, 2018]{Supervisor: Prof. Ken Jackson }


\begin{document}

\frame{\titlepage}
\begin{frame}
    \frametitle{Overview}
    \tableofcontents
\end{frame}


%---------------------------------------------------------
\section{Monte Carlo Simulation}
%---------------------------------------------------------


\begin{frame}
\frametitle{MC - Monte Carlo Simulation}
\begin{enumerate} 
    \item To simulate some unknown value $\mu$, we interpret $\mu$ probabilistically, i.e. expresses it as the expected value of a function of some random variable $X$
    \[
        \mu = \E[\bX \sim p(\bX)]{f(\bX)}
    \]
    \item Sample $X_1, \cdots, X_n \overset{i.i.d.}{\sim} p$ , and compute the Monte Carlo (MC) estimator
    \[
        \hat{\mu} = \frac{1}{n} \sum_{i=1}^n f(\bX_i)
    \]
\end{enumerate}    
\end{frame}


% explain LLN:
%     for any nonzero margin specified, no matter how small, with a sufficiently large sample there will be a very high probability that the average of the observations will be close to the expected value; that is, within the margin.


\begin{frame}{MC - Variance Reduction}
We evaluate an estimator by computing
\begin{align*}
    \E{\hat{\mu}} &= \frac{1}{n} \sum_{i=1}^n \E{f(\bX_i)} = \mu \quad \text{(unbiased)} \\ 
     \var{\hat{\mu}} &= \frac{1}{n^2}\sum_{i=1}^n \var{f(\bX_i)} = \frac{\sigma^2}{n}
\end{align*}
where $\E{\bX_i} = \mu$ and $\var{\bX_i}=\sigma^2$. Intuitively, estimator is more precise, i.e. smaller CI, with decreased variance. The strategies and trade-offs for variance reduction varies
\begin{enumerate}
    \item Increase sample size $n$ comes at increased runtime cost
    \item Decrease random variable's variance, the tricks!
\end{enumerate}
\end{frame}





\begin{frame}{MC - Importance Sampling}
\textbf{Motivation} In the case of estimating rare events, we want to compute $\mu = \E{f(\bX)}$ where $f(\bX)$ is close to zero out side of a region $\mathcal{D}$. We can exploit this to do variance reduction with \textbf{importance sampling}
\begin{enumerate}
    \item Find an alternate \textit{proposal distribution} to sample $X_1, \cdots, X_n \sim q(\bX)$ such that the \textbf{variance is reduced}
    \[
        \var[\bX \sim q]{\frac{f(\bX)p(\bX)}{q(\bX)}} < 
        \var[\bX \sim p]{f(\bX)}
    \]
    \item Adjust the estimates to account for sampling from the $q(\bX)$ such that the estimator remains \textbf{unbiased}
    \begin{align*}
    \mu = \E[\bX \sim p]{f(\bX)} 
    &= \int_{\mathcal{D}} f(\bx) p(\bx) d\bx \\
    &= \int_{\mathcal{D}} \frac{f(\bx)p(\bx)}{q(\bx)} q(\bx) d\bx 
    = \E[\bX \sim q]{\frac{f(\bX)p(\bX)}{q(\bX)}}
\end{align*}
\end{enumerate}
\end{frame}

%---------------------------------------------------------
\section{Problem Setup}
%---------------------------------------------------------
 
\begin{frame}{Problem Setup - Finance Terminologies}
\begin{enumerate}
    \item \textbf{Portfolio} A collection of investments
    \item \textbf{Credit Risk} Risk of default in a portfolio
    \item \textbf{Systemic Risk} Overall risk that affects all assets, i.e. government policy, international economic forces, or acts of nature.
    \item \textbf{Idiosyncratic Risk} Risk endemic to a particular asset, i.e. aspects of a company
\end{enumerate}
\end{frame}



\begin{frame}{Problem Setup - A Credit Risk Model}
\textbf{Guassian Copula Factor Model} is a model that measure credit risk by modeling a set of underlying risk factors as Gaussian random variables. Let 
\begin{enumerate}
    \item $Y_k$ be the default indicator for the $k$th borrower, i.e. 1 if $k$th borrower defaults and 0 otherwise. Given systematic risk factors $Z_i \sim \normal{0,I_S}$ and idiosyncratic risk factors $\epsilon_k \sim \normal{0,I_N}$, we have
    \[
        Y_k = \mathbbm{1}_{X_k > x_k} \qquad \text{$x_k$ picked to match $p_k$}
    \]
    where $l$ is the tail probability supplied, and $X_k$ is a weighted sum of the risk factors
    \[
        X_k = \boldsymbol{\beta}_k^T \matr{Z}  + \sqrt{1 - \boldsymbol{\beta}_k^T \boldsymbol{\beta}_k} \epsilon_k
    \]
    \item $p_k$ be probability of default for $k$th borrower
    \item $c_k$ be loss resulting from default of $k$th borrower 
    \item $L$ be total loss from defaults
    \[
        L = c_1 Y_1 + \cdots + c_N Y_N    
    \]
\end{enumerate}
\end{frame}


\begin{frame}{Problem Setup - MC+IS applied to Risk Model}
We are interested in the probability that a given portfolio of lenders will result in a loss greater than a specified amount. And we use Monte Carlo simulation to estimate this probability
\begin{align*}
    P(L \geq l) 
    &= \E[\bZ \sim \normal{0, I_S}\,\, Y_k \sim p_k]{\mathbbm{1}_{L \geq l}}
\end{align*}
\end{frame}


\end{document} 