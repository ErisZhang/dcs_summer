\documentclass[11pt]{article}
\input{\string~/.preamble}

\begin{document}


% arg1=pdfurl arg2=pagenum arg3=sectiontitle
\newcommand{\linksection}[3][2018_an_importance_sampling_scheme_for_multi_credit_state_portfolios.pdf]{
    \subsection*{\href[page=#2]{#1}{#3}}
}

\renewcommand{\norm}[1]{\left\lVert#1\right\rVert}
\renewcommand{\E}[2][]{\mathbb{E}_{#1}\left\{#2\right\}}
\newcommand{\var}[2][]{var_{#1}\left\{#2\right\}}
\newcommand{\cov}[1]{cov\{#1\}} 
\newcommand{\normal}[1]{\mathcal{N}\left(#1\right)}
\newcommand{\exponents}[1]{exp\left\{#1\right\}}
\newcommand{\indicator}[1]{\mathbbm{1}_{#1}}

\newcommand{\bmu}{\boldsymbol{\mu}}
\newcommand{\bpi}{\boldsymbol{\pi}}
\newcommand{\bTheta}{\boldsymbol{\Theta}}
\newcommand{\bSigma}{\boldsymbol{\Sigma}}
\newcommand{\bphi}{\boldsymbol{\phi}}

\newcommand{\calA}{\mathcal{A}}
\newcommand{\calL}{\mathcal{L}}
\newcommand{\calE}{\mathcal{E}}
\newcommand{\calR}{\mathcal{R}}
\newcommand{\calC}{\mathcal{C}}
\newcommand{\calD}{\mathcal{D}}
\newcommand{\bx}{\matr{x}}
\newcommand{\bt}{\matr{t}}
\newcommand{\bw}{\matr{w}}
\newcommand{\bX}{\matr{X}}
\newcommand{\bZ}{\matr{Z}}
\newcommand{\bz}{\matr{z}}
\newcommand{\bu}{\matr{u}}



\subsection*{Terminologies}

\begin{enumerate}
    \item \textbf{Portfolio} a collection of investment
    \item \textbf{Security} a tradable financial asset.
    \item \textbf{Bond} an instrument of indebtedness of the bond issuer to the holders. The issuer owes the holders a debt and is obliged to pay them interest (coupon) or repay the principal at a later date, termed maturity date.
    \begin{enumerate}
        \item \textbf{Holder} lender/creditor 
        \item \textbf{Issuer} borrower/debtor
    \end{enumerate}
    Bond is a type of security
    \item \textbf{Default} is the failure to meet legal obligations of a loan, i.e. a corporation or government failing to pay a bond which has reached maturity
    \item \textbf{Credit Risk} the risk of default on a debt that may arise from a borrower failing to make required payments 
    \item \textbf{Credit Exposure} is the total amount of credit made available to a borrower by a lender. Indicates extent to which the lender is exposed to risk of loss in event of borrower's default. 
    \item \textbf{Loss Given Default (LGD)} the share/percentage of an asset that is lost if a borrower defaults
    \item \textbf{Exposure at Default (EAD)} is the immediate loss that the lender would suffer if the borrower fully defaults
    \item \textbf{Probability of Default (PD)} likelihood of a default over a particular time horizon, i.e. an estimate of likelihood that a borrower will be unable to meet its debt obligations. PD is derived by analyzing obligor's capacity to repay debt, and is associated with characteristics such tas inadequate cash flow, declining revenue, inability to implement a business plan
    \item \textbf{Expected loss} sum of values of all possible losses, each multiplied by the probability of that loss occurring.
    \item \textbf{Systematic Risk} overall risk that affects all assets like fluctuations in the stock market or interest rates
    \item \textbf{Idosyncratic Risk} risk that is endemic to a particular asset and not a whole investment portfolio. Idiosyncratic risk is generally less predictable. Diversification of an investment portfolio eliminates it somewhat
\end{enumerate}


\begin{defn*}
    \textbf{Exponential Twisting} distribution shifting technique used in rare-event simulation. For a distribution with density $f$, and parameter $\theta$, the exponentially twisted distribution is defined as follows
    \[
        f_{\theta}(x) = \frac{e^{\theta X} f(x)}{\E[f]{e^{\theta X}}}
    \]
    \begin{enumerate}
        \item for $f = \normal{\mu, \sigma^2}$, $f_{\theta} = \normal{\mu + \theta\sigma^2, \sigma^2}$, i.e. effectively shifts the mean of the Gaussian
    \end{enumerate}
    Used in importance sampling as an alternative distribution that reduces estimate's variance
\end{defn*}


\end{document}
