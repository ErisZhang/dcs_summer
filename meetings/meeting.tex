\documentclass[11pt]{article}
\input{\string~/.preamble}

\begin{document}


\renewcommand{\norm}[1]{\left\lVert#1\right\rVert}
\renewcommand{\E}[2][]{\mathbb{E}_{#1}\left\{#2\right\}}
\newcommand{\var}[2][]{var_{#1}\left\{#2\right\}}
\newcommand{\cov}[1]{cov\{#1\}} 
\newcommand{\normal}[1]{\mathcal{N}\left(#1\right)}
\newcommand{\exponents}[1]{exp\left\{#1\right\}}
\newcommand{\indicator}[1]{\mathbbm{1}_{#1}}

\newcommand{\bmu}{\boldsymbol{\mu}}
\newcommand{\bpi}{\boldsymbol{\pi}}
\newcommand{\bTheta}{\boldsymbol{\Theta}}
\newcommand{\bSigma}{\boldsymbol{\Sigma}}
\newcommand{\bphi}{\boldsymbol{\phi}}

\newcommand{\calA}{\mathcal{A}}
\newcommand{\calL}{\mathcal{L}}
\newcommand{\calE}{\mathcal{E}}
\newcommand{\calR}{\mathcal{R}}
\newcommand{\calC}{\mathcal{C}}
\newcommand{\calD}{\mathcal{D}}
\newcommand{\calZ}{\mathcal{Z}}
\newcommand{\bx}{\matr{x}}
\newcommand{\bt}{\matr{t}}
\newcommand{\bw}{\matr{w}}
\newcommand{\bX}{\matr{X}}
\newcommand{\bZ}{\matr{Z}}
\newcommand{\bz}{\matr{z}}
\newcommand{\bu}{\matr{u}}



\begin{enumerate}
    \item \textbf{The risk modeling} \\
    Given 
    \[
        y_n = \beta_n^T \calZ + \sqrt{1 - \beta_n^T \beta_n} \epsilon_n
    \]
    where $\calZ \sim \normal{0, I_S}$ and $\epsilon \sim \normal{0, I_N}$, the resulting $y_n \sim \normal{0,1}$ as shown below 
    \[
        \E{y_n} = \sum_{i} \beta_{n,i} \E{\calZ_i} + \sqrt{1 - \beta_n^T \beta_n} \E{\epsilon_n} = 0
    \]
    \begin{align*}
        \var{y_n} 
        &= \E{(y_n - \E{y_n})^2} \\
        &= \E{ (\beta_n^T \calZ + \sqrt{1 - \beta_n^T \beta_n} \epsilon_n)^2 }\\
        &= \beta_n^T \beta_n \E{(\calZ - 0)^2} + (1-\beta_n^T \beta_n) \E{(\epsilon_n - 0)^2} \\
        &= \beta_n^T \beta_n \var{\calZ} + (1-\beta_n^T \beta_n) \var{\epsilon_n} \\
        &= 1
    \end{align*}
    \item \textbf{Motivation for threshold between different states $H_{c(n)}^c$} \\
    The motivation is to model discrete probability in the credit state matrix with a continuous distribution such as the gaussian. In this cawe, we want to set $H_{c(n)}^c$ such that
    \[
        p(H_{c(n)}^{c-1}\leq y_n \leq H_{c(n)}^c) = p_{c(n)}^c
    \]
    therefore we can write 
    \[
        p(y_n \leq H_{c(n)}^c) = \sum_{\gamma = 1}^c p_{c(n)}^{\gamma}
        \qquad 
        \overset{y_n \sim \normal{0,1}}{\longrightarrow} 
        \qquad 
        \Phi(H_{c(n)}^c) = \sum_{\gamma = 1}^c p_{c(n)}^{\gamma}
    \]
    \item \textbf{Confidence interval for monte carlo estimation}
    \[
        p(L_N(\calZ, \epsilon)\geq l) \in 
        p(L_N(\calZ, \epsilon)\geq l) \pm CI
    \]
    Idea is that for the two naive algorithms that are purported to be equivalent, the CI should be approximately the same
\end{enumerate}



\begin{enumerate}
    \item \textbf{Likelihood Function} Note 
    \[
        Z \sim \normal{0, I}    
        \qquad \leftarrow \qquad 
        Z \sim \normal{\mu, I}
    \]
    The quotient is then 
    \[
        \frac{\sqrt{2\pi}^S e^{-Z^TZ}{2}}{\sqrt{2\pi}^S e^{-(Z-\mu)^T(Z-\mu)}{2}}
    \]
\end{enumerate}

\end{document}
 